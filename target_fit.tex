\documentclass[12pt,letterpaper]{article}
\usepackage[utf8]{inputenc}
\usepackage{amsmath}
\usepackage{amsfonts}
\usepackage{amssymb}
\usepackage{graphicx}
\usepackage[left=2cm,right=2cm,top=2cm,bottom=2cm]{geometry}
\usepackage{tcolorbox}
\author{Orlando Soto}

\usepackage{epstopdf}	% Included EPS files automatically converted to PDF to include with pdflatex
%\epstopdfsetup{update}
\epstopdfDeclareGraphicsRule
{.gif}{png}{.png}{convert #1 \OutputFile}
%%{.gif}{png}{.png}{C:/"Program Files"/ImageMagick-6.8.0-Q16/convert.exe #1 \OutputFile} %windows
\DeclareGraphicsExtensions{.pdf, .jpeg, .jpg, .gif, .png}
\graphicspath{{.}}	% Root directory of the pictures 


\begin{document}
Consider the reconstruction of the position of the vertex of an interaction.
\begin{itemize}
\item If the target were just a point, of infinitesimal volume, this position will be reconstructed as a Gaussian function, due to the big number of random variables that are being summed (central limit theorem).
\item In reality the target doesn't has differential volume, then it doesn't has differential dimensions.
\end{itemize}
The reconstructed position will be then a Gaussian convoluted with the spatial function describing the target.

Let's focus on z position. Let be $f_t(z)$ the function that describes the target spatial function, $f_m(z)$ the function that describes the procedure of measurement and $v(z)$ the reconstructed vertex function $v(z) = f_t \star f_m$.

first point:
\begin{align}
f_t(z) &= \delta (z-m) \label{tf_1}\\
f_m(z) &= \frac{1}{\sqrt{2\sigma^2}}e^{\frac{-z^2}{2\sigma^2}}\label{mf_1}
\end{align}
\begin{align}
v(z) &= \int_{-\infty}^{\infty} f_t(z-t)f_m(t)dt\label{rvf_1}\\
v(z) &= \int_{-\infty}^{\infty} \delta(z-t-m)f_m(t)dt\\
v(z) &= f_m(z-m) = \frac{1}{\sqrt{2\sigma^2}}e^{\frac{-(z-m)^2}{2\sigma^2}}\\
\end{align}

Considering the real target as a step function:
\begin{equation}
f_t(z) = \left\lbrace\begin{matrix}
1,~~~|z-m|<\frac{w}{2}\\
0,~~~ o.w.
\end{matrix}\right. \label{tf_2}
\end{equation}
using \eqref{rvf_1} and \eqref{tf_2}.
\begin{align}
v(z) &= \int_{-\infty}^{\infty} f_t(z-t)f_m(t)dt\label{rvf_2}\\
v(z) &= \int_{m-z-w/2}^{m-z+w/2} f_m(t)dt\\
v(z) &= \int_{m-z-w/2}^{m-z+w/2} \frac{1}{\sqrt{2\sigma^2}}e^{\frac{-t^2}{2\sigma^2}}dt\label{vz_2b}
\end{align}

considering the function $erf$

\begin{equation}
erf(t) = \frac{2}{\sqrt{\pi}}\int_0^t e^{-x^2}dx \label{erf}
\end{equation}
using \eqref{erf} and \eqref{vz_2b}:
%
\begin{equation}
v(z) = \frac{1}{2}\left(erf(\frac{m-z+w/2}{\sqrt{2}\sigma} ) - erf(\frac{m-z-w/2}{\sqrt{2}\sigma} )\right)
\end{equation}
Adding a second order polynomial and fitting on Carbon and Deuterium events gives:
%
\begin{figure}[!ht]
\includegraphics[width=0.9\textwidth]{CD_D_fit_pol2.gif}
\caption{Fitting on Deuterium target using C and D2 data.}
\end{figure}
%
\begin{figure}[!hb]
\includegraphics[width=0.9\textwidth]{CD_C_fit_pol2.gif}
\caption{Fitting on Carbon target using C and D2 data.}
\end{figure}
%
As expected, the dispersion due to vertex measurement ($\sigma$ on $f_m(z)$) are very close considering the estimations using each of the targets.

\tcbox{
D2: $\sigma = 0.2093$


C: $\sigma= 0.2065$
}

\end{document}